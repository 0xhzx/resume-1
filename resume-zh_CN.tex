% !TEX TS-program = xelatex
% !TEX encoding = UTF-8 Unicode
% !Mode:: "TeX:UTF-8"

\documentclass{resume}
\usepackage{zh_CN-Adobefonts_external} % Simplified Chinese Support using external fonts (./fonts/zh_CN-Adobe/)
%\usepackage{zh_CN-Adobefonts_internal} % Simplified Chinese Support using system fonts
\usepackage{linespacing_fix} % disable extra space before next section
\usepackage{cite}
\usepackage[colorlinks,linkcolor=blue]{hyperref}

\begin{document}
\pagenumbering{gobble} % suppress displaying page number

\name{侯建鹏 }

% {E-mail}{mobilephone}{homepage}
% be careful of _ in emaill address
\contactInfo{houjp1992@gmail.com}{(+86) 152-0144-2067}{ \url{https://github.com/houjp/} }
%\vspace{-1ex}
\centerline{求职意向: 机器学习算法工程师}
% {E-mail}{mobilephone}
% keep the last empty braces!
%\contactInfo{xxx@yuanbin.me}{(+86) 131-221-87xxx}{}

%\vspace{-1ex}
 
\section{\faGraduationCap\  教育背景}
\datedsubsection{\textbf{中国科学院计算技术研究所(保送)}~~ \ 硕士, 计算机软件与理论, 排名: 5\%}{2014 -- 2017}

\datedsubsection{\textbf{北京科技大学}~~ \ 学士, 计算机科学与技术, 排名: 4/124}{2010 -- 2014}
\vspace{1ex}

\section{\faUsers\ 项目经历}
\datedsubsection{\textbf{BDA大数据分析系统} }{2015.10 -- 2016.03}
%\role{实习}{经理: 高富帅}
%xxx后端开发
\begin{itemize}
  \item 作为 BDA Lib 核心开发者,在分布式计算框架 Spark 上实现了 CART/GBDT/GBRT/RF 算法的 分布式版本,同时完成了这些算法单机版本的开发。
  \item 实现了 BDA Studio 中需要的数据挖掘基本组件 (包括特征索引、特征合并、特征选择、特征值 归一化、模型输出结果评分等功能),并构建真实场景的应用实例。
\end{itemize}

\vspace{-1.5ex}

\datedsubsection{\textbf{中国电信大数据竞赛 (1st/1112; CNY 200,000)}}{2015.12 -- 2016.03}
%\role{Golang, Linux}{个人项目,和富帅糕合作开发}
\begin{onehalfspacing}
根据海量用户行为数据(大于四亿条,25.38G),预测用户接下来一周对十个视频网
站的访问量。
\begin{itemize}
  \item 基于 Spark 分布式计算框架,提出并实现了 GBLR 多目标回归算法,使得 F1 值提升 0.8\%。
  \item 提出并实现了用于用户分类的概率排序模型,使得 F1 值提升 0.6\%。
\end{itemize}
\end{onehalfspacing}

\vspace{-1.5ex}

\datedsubsection{\textbf{SIGHAN-2015 Chinese Spelling Check Task}}{2015.03 -- 2015.05}
%\role{\LaTeX, Python}{个人项目}
\begin{onehalfspacing}
该评测任务是对繁体中文进行拼写检查,给出正确的拼写结果。
\begin{itemize}
  \item 实现候选生成排序模型:根据同音、近音、形近字,为句中每一个繁体字生成候选, 并打分排序。
  \item 实现两轮候选重排序模型:采用简单特征进行预排序,在第一次排序的基础上结合复杂特征进行第二次排序,以此提高排序效率。最终取得第一名,F值超出第二名18\%。
\end{itemize}
\end{onehalfspacing}

% Reference Test
%\datedsubsection{\textbf{Paper Title\cite{zaharia2012resilient}}}{May. 2015}
%An xxx optimized for xxx\cite{verma2015large}
%\begin{itemize}
%  \item main contribution
%\end{itemize}

\section{\faSitemap\ 实习经历}
\datedsubsection{\textbf{滴滴研究院~智能平台部}}{2016.07 -- 2016.08}
%\role{\LaTeX, Python}{个人项目}
\begin{onehalfspacing}
%优雅的 \LaTeX\ 简历模板, https://github.com/billryan/resume
\begin{itemize}
  \item 完成专车调度系统开发、测试并上线,实现对城市各区域专车供需不平衡状态的动态调整。
  \item 基于Kafka及MySQL完成专车调度日志系统重构,解耦子模块间相互依赖,提高系统可用性。
\end{itemize}
\end{onehalfspacing}

\vspace{-1.5ex}

\datedsubsection{\textbf{百度~网页搜索部}}{2013.12 -- 2014.05}
%\role{\LaTeX, Python}{个人项目}
\begin{onehalfspacing}
\begin{itemize}
  \item  完成 livewdbbroom 工具多线程开发,通过测试并上线,效率提升 2 倍以上。
  \item 完成离线时效性日志体系建设及离线时效性问题追查平台建设,并上线。
\end{itemize}
\end{onehalfspacing}


\section{\faHeart\ 学术竞赛}
\datedline{\textit{冠军 (1st/1112; CNY 200,000)}~中国电信大数据算法应用大赛}{2016.03}
\datedline{\textit{冠军}~SIGHAN-2015 Chinese Spelling Check Task}{2015.06}
\datedline{\textit{冠军}~全国大学生计算机博弈大赛}{2013.08}
\datedline{\textit{银牌 (17th/200)}~ACM-ICPC 国际大学生程序设计竞赛亚洲区域赛北京赛区}{2015.11}
\datedline{\textit{四强 (4th/2293)}~阿里巴巴大数据竞赛“新浪微博互动预测大赛”}{2015.12}
\datedline{\textit{一等奖}~华北五省及港澳台大学生计算机应用大赛}{2013.11}
\datedline{\textit{一等奖}~全国软件专业人才设计与创业大赛北京赛区}{2012.04}



\section{\faCogs\ 个人能力}
% increase linespacing [parsep=0.5ex]
\begin{itemize}[parsep=0.5ex]
  \item 熟悉 Scala、C++、Shell,熟悉基本数据结构和算法,有良好的编程风格。
  \item 有丰富的基于 Spark 分布式计算框架的机器学习算法开发经验。
  \item 熟悉数据挖掘、机器学习领域基本算法。
\end{itemize}

%\section{\faInfo\ 其他}
%% increase linespacing [parsep=0.5ex]
%\begin{itemize}[parsep=0.5ex]
%  \item 技术博客: http://blog.yours.me
%  \item GitHub: https://github.com/username
%  \item 语言: 英语 - 熟练(TOEFL xxx)
%\end{itemize}

%% Reference
%\newpage
%\bibliographystyle{IEEETran}
%\bibliography{mycite}
\end{document}
